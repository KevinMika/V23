\section{Theorie}
Im Grunde beschäftigt sich der Versuch nicht mit quantenmechanischen,
sondern mit akustischen Phänomenen. Es lassen sich im Hinblick auf die mathematische
Beschreibung akustischer Phänomene viele Analogien zur Wellenmechanik der Quantentheorie finden.
Diese Analogien sind nicht exakt, da dies schon an den unterschiedlichen Dispersionsrelationen scheitert,
führen oft aber weit genug, um Charakteristika der quantenmechanischen Beschreibung
eines Teilchens im klassischen Regime der Akustik beobachten zu können.
\subsection{Gemeinsamkeiten von Akustischen und Quantenmechanischen Systemen}
Aus der linearen Euler Gleichung
\begin{equation}
  \frac{\partial \nu}{\partial t}=-\frac{1}{\rho}\text{grad}(p)
\end{equation}
und der Kontinuitätsgleichung folgt, dass sich eine eindimensionale Welle mit
\begin{equation}
  p(x)=p_0\text{cos}(kx-\omega t)
\end{equation}
beschreiben lässt. In einer Röhre mit der Länger $L$ lässt sich mit den Randbedingungen herleiten, dass
$k=\frac{n\pi}{L}$ ist.\\
Analog kann damit ein Teilchen in einem Potentialtopf beschrieben werden. Das Teilchen
wird mit der Schrödingergleichung beschrieben:
\begin{equation}
  E\psi(r)=-\frac{\hbar}{2m}\increment\psi(r)+V(r)\psi(r).
\end{equation}
In einem Unendlich hohen Potentialtopf dessen Zwischenraum ein Potential von $V=0$ besitzt,
vereinfacht sich die Schrödingergleichung zu
\begin{equation}
  E\psi(r)=-\frac{\hbar}{2m}\increment\psi(r)
\end{equation}
dessen Lösung für die Eigenwerte Wellen der Form
\begin{equation}
  \psi(x)=A\text{sin}(kx).
\end{equation}
Aus den Randbedingungen folgt $k=\frac{n\pi}{L}$. Die Eigenwerte sind
\begin{equation}
  E(k)=\frac{\hbar^2 k^2}{2m}=\frac{\hbar^2n^2\pi^2}{2mL^2}.
\end{equation}
Die gemeinsamkeit, dass beide Wellenfunktionen ein delokalisiertes Objekt beschreiben.
Ebenfalls können bei beiden Systemen stehende Wellen auftreten.
\subsection{Unterschiede von Akustischen und Quantenmechanischen Systemen}
Die Wellengleichungen unterscheiden sich in der zeitlichen Ableitung.
Wärend die klassische Wellengleichung auf Grund der zweiten Zeitableitung periodische Lösungen besitzt, folgt
dies im quantenmechanischen Fall aus der ersten Zeitableitung in Verbindung mit einem
komplexen Phasenfaktor. Komplexwertige Funktionen können nicht direkt beobachtet werden.
Gemessen werden kann nur das Betragsquadrat der Lösung und diese nur in statistischen
Auswertungen.\\
Die Dispersionsrelationen unterscheiden sich, dass in der klassischen Mechanik linear ist,
während Materiewellen parabolischen Dispersionsrelationen folgt. Die Folge daraus sind
unterschiedliche Gruppen- und Phasengeschwindigkeit.\\
Im unendlichen Potentialtopf wird ein ein Verschwinden der Wellenfunktion an den Rändern
gefordert, während der Druck dort einen Schwingungsbauch vorweist.
\subsection{Analogien zur Festkörperphysik}
Über eingebaute Blenden zwischen den Röhren, welche den Schall streuen, kann eine Elektronenwelle
welche an einem Atom gestreuet wird, simuliert werden. Dies führt zu neuen Resonanzfrequenzen
welche zur Bandstruktur führt, wobei die Blendgröße (Irisdurchmesser) ein Maß für den
Wirkungsquerschnitt ist. Die Bandstruktur bilden sich, wenn die
Bragg Bedingung erfüllt wird, also
\begin{equation}
  n\cdot\lambda=2\cdot a
\end{equation}
gilt. $n$ ist Element der natürlichen Zahlen und $a$ ist der Abstand der reflektierenden Ebenen. Im
diesem Versuch stellen die Blenden die reflektierenden Ebenen dar und der Abstand ist dann die Röhrenlänge.
