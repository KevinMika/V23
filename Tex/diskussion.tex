\section{Diskussion}
Der experimentell bestimmte Wert für die Schallgeschwindigkeit
$c=\SI{342.098}{\meter\per\second}$ weicht nur $0,26\%$ vom Literaturwert ab,
was eine ziemlich präzise Messung darstellt.\\
Abbildungen \ref{fig:plot1}, \ref{fig:plot2}, \ref{fig:plot4} und \ref{fig:plot10} zeigen gut die Analogie
zwischen den Röhren und dem Quantenmechanischen Potentialtopf, welcher
nur diskrete Werte als Eigenenergie zulässt. Die Resonanzfrequenzen sind
diskret und deren Anzahl nimmt mit zunehmender Länge der Röhre zu.\\
In Abbildung \ref{fig:plot6}, \ref{fig:plot12}, \ref{fig:plot13}, \ref{fig:plot14} und \ref{fig:plot11} ist gut zu sehen, wie sich eine Welle bzw. die Wellenfunktion
eines Elektrons verhält, wenn es an einem hindernis mit variierendem Wirkungsquerschnitt
gestreut wird verhält, simuliert mit Iriden zwischen den Rohrabschnitten mit verschiedenen
durchmessern. Die Bandstrukturen in Abbildung \ref{fig:plot7} und \ref{fig:plot15}, welche aus der Streuung
resultieren zeigen die Energieniveaus des Elektrons.\\
Die Anzahl der Resonanzen pro Band nimmt mit steigender Anzahl von Elementarzellen
(simuliert mit $\SI{50}{\milli\meter}$ Röhren) zu, wie in Abbildung \ref{fig:plot8} zu sehen ist.\\
Der Einfluss des Raumes zwischen den Atomen, kann mit verschieden langen Rohren zwischen den
Iriden beeinflusst werden: Die Resonanzfrequenzen verschieben sich, während die
Anzahl der Maxima bleibt gleich (Abbildung \ref{fig:plot9}).\\
Ein Molekül, bestehend aus $2$ Atomen, wurde in Abbildung \ref{fig:plot16} simuliert mit alternierenden Iridendurchmesser (Wirkungsquerschnitt).
Verglichen mit einem Spektrum mit einem konstanten Irisdurchmesser wird deutlich, wie sich die Resonanzfrequenzen und damit
auch die Bandstrukturen aufspalten in Zwischenstufen, was in Abbildung \ref{fig:plot17} anschaulich dargestellt wird.\\
Der Einbau eines Defekts führte zu einer Störung der Bandstruktur, was zur bildung
eines neuen Zustands führt (Abbildung \ref{fig:plot18} und \ref{fig:plot19}).\\
Die nutzung von Schallwellen als Analogon zur Wellenfunktion eines Elektrons eignet sich gut,
um das Verhalten eines jenen Elektrons in einem Gitter oder Potentialtopfes zu untersuchen.
